% !TEX encoding = UTF-8 Unicode
\documentclass[10pt,brazil]{beamer}
\usepackage[utf8]{inputenc}
\usepackage[brazil]{babel}
\usepackage{color,colortbl,multirow}
\usetheme{Luebeck}
\DeclareUnicodeCharacter{22EF}{ }
\usepackage{gensymb}
\usepackage{hyperref}
\setbeamertemplate{footline}
{
	\leavevmode%
	\hbox{%
		
		\begin{beamercolorbox}[wd=.666666\paperwidth,ht=2.25ex,dp=1ex,center]{title in head/foot}%
			\usebeamerfont{title in head/foot}\inserttitle
		\end{beamercolorbox}%
		\begin{beamercolorbox}[wd=.333333\paperwidth,ht=2.25ex,dp=1ex,right]{date in head/foot}%
			\usebeamerfont{date in head/foot}\insertshortdate{}\hspace*{2em}
			\insertframenumber{} / \inserttotalframenumber\hspace*{2ex} 
	\end{beamercolorbox}}%
	\vskip0pt%
}

\usepackage{caption}
\definecolor{azul}{HTML}{962038}
\setbeamercolor{structure}{fg=azul}
\usepackage{multicol}
\usepackage{media9}


\usepackage[alf]{abntex2cite}
\pgfdeclareimage[height=1.8cm]{logo}{UFOP}
\logo{\pgfuseimage{logo}}
\newcommand{\nologo}{\setbeamertemplate{logo}{}}
\setlength{\parskip}{0.2cm}

\title[Resumo]{Andamento do Trabalho Prático}

\author[ICEA - UFOP]{Guilherme Marx, Ícaro Bicalho e Leonardo Sartori}

\begin{document}
 	   \maketitle
		\begin{frame}{Sumário}
			\tableofcontents
		\end{frame}
%---------------------------------------------------------


\section{Adequação ao Cronograma}

\begin{frame}{Adequação ao cronograma}	
	\begin{itemize}
		\item Na apresentação anterior, foi proposto o seguinte cronograma:
		\begin{enumerate}
			\item De 14/09 à 13/10: \hfill Desenvolvimento do módulo Model.
			\item De 14/10 à 13/11: \hfill Desenvolvimento do módulo View.
			\item De 14/11 à 09/12: \hfill Desenvolvimento do módulo Controller.
		\end{enumerate}
		\item O mesmo não pôde ser seguido, isso devido ao fato de que o \emph{Middleware} que iríamos utilizar não serviu ao nosso propósito e, após muitas pesquisas, vimos a inviabilidade de continuar usando arquitetura \emph{Peer-To-Peer} nesse projeto.	
	\end{itemize}
\end{frame}

\section{Decisões de Projeto}
\begin{frame}{Arquitetura de Rede}
	\begin{itemize}
		\item Abandono da arquitetura \emph{Peer-To-Peer} e adoção da arquitetura Cliente-Servidor.
		\item Como a ideia do trabalho é não ter um servidor centralizado, o projeto foi dividido em duas entidades:
		
		\begin{enumerate}
			\item \emph{Banco de IDs}: Um mini-servidor, serve para armazenar IDs das operações feitas no Banco de Dados e o nome do respectivo cliente que possui os scripts dessa alteração.
			\item \emph{Cliente}: Apesar do nome, ele é cliente em sua relação com o Banco de IDs e também pode ser cliente ou servidor em sua relação com outros Clientes.
		\end{enumerate}
		\item Comunicação em sua maioria é feita por meio de duas interfaces RMI, uma entre Cliente e Banco de IDs e outra entre Cliente e Cliente.
	\end{itemize}
\end{frame}

\begin{frame}{A comunicação RMI}


{\small
		\begin{enumerate}
			\item Cliente que deseja fazer uma alteração no banco de dados local se comunica com o Banco de IDs para que este atualize as entradas em sua tabela de IDs colocando o nome deste Cliente no respectivo local da tabela.

			\item Todos os clientes possuem uma \emph{Thread} que monitora essa tabela via RMI, assim, no momento em que essa \emph{Thread} detecta uma alteração na tabela, os Clientes pegam o nome de quem realizou essa alteração através de um método remoto. 

			\item Uma vez em posse do nome de quem fez a alteração, os Clientes desatualizados chamam um método remoto no Cliente atualizado e pegam o \emph{script} da alteração.

			\item Ao aplicar o mesmo \emph{script} em seus respectivos Bancos de Dados locais, os bancos se mantém atualizados, a replicação foi feita com sucesso.
		\end{enumerate}
		}
\end{frame}


\begin{frame}{A arquitetura MVC}
	\begin{itemize}
	 \item Para facilitar a organização das responsabilidades, decidimos por atribuir ao pacote \emph{Model} as funções de Coordenar alterações no banco de dados e comunicar via RMI com o Banco de IDs e com os outros Clientes.
	 \item Assim, fica claro que o pacote \emph{Model} concentra a maior parte do trabalho, o que deixa o cronograma inicial (com 1 mês para o desenvolvimento de cada módulo) obsoleto.
	 \item Por esse motivo, não pudemos trazer o pacote \emph{Model} completo para apresentação, como havia sido previsto.
	 \item Decidimos, então, elaborar um projeto de teste demonstrando a forma que ocorrerá a comunicação RMI do projeto, isso será apresentado ao final dessa apresentação.
	\end{itemize}
\end{frame}


\section{O que já foi feito}
\begin{frame}{O que já foi feito}
	\begin{enumerate}
		\item Diagrama Entidade-Relacionamento do Banco de Dados;
		\item Esquema Relacional (projeto lógico) do Banco de Dados;
		\item \emph{Script} de criação do Banco de Dados;
		\item Interfaces RMI de todas as formas de comunicação do projeto;
		\item Objetos Model e Objetos DAO dos Clientes.
	\end{enumerate}
\end{frame}	

\section{Dúvidas?}
\begin{frame}{Dúvidas?}
Dúvidas, críticas, sugestões?
\end{frame}

\section{Finalização}

\begin{frame}{Repositório GitHub do projeto}
	
	\href{https://github.com/guilhermemarx14/Concessionaria}{https://github.com/guilhermemarx14/Concessionaria}
	
\end{frame}


	\end{document}
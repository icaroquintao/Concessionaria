% !TEX encoding = UTF-8 Unicode
\documentclass[brazil, a4paper,12pt]{article}
\bibliographystyle{plain}
\usepackage[brazil]{babel}
\usepackage{graphicx}
\usepackage{geometry}
\usepackage[utf8]{inputenc}
\usepackage[T1]{fontenc}
\usepackage{listings}
\usepackage{indentfirst}
\usepackage{lmodern}
\geometry{a4paper,left=3cm,right=3cm,top=2.5cm,bottom=2.93cm}

\lstset{numbers=left,
stepnumber=0,
firstnumber=1,
numberstyle=\tiny,
extendedchars=true,
breaklines=true,
frame=tb,
tabsize=2,
basicstyle=\footnotesize,
stringstyle=\ttfamily,
showstringspaces=false
}

\renewcommand{\lstlistingname}{Programa}

\begin{document}
\begin{titlepage}

  \vfill

  \begin{center}
    \begin{large}
      Universidade Federal de Ouro Preto
    \end{large}
  \end{center}

  \begin{center}
    \begin{large}
      Instituto de Ciências Exatas e Aplicadas
    \end{large}
  \end{center}

  \begin{center}
    \begin{large}
      Departamento de Computação e Sistemas
    \end{large}
  \end{center}

  \vfill

  \begin{center}
    \begin{Large}
      \textbf{SISTEMAS DISTRIBUÍDOS\\[0.4cm] 
        Resumo - Andamento do Trabalho Prático}               
    \end{Large}
  \end{center}


  \vfill

  \begin{center}
    \begin{large}

   	Guilherme Marx Ferreira Tavares \hfill 14.1.8006

	Ícaro Bicalho Quintão \hfill 14.1.8083
	
	Leonardo Sartori de Andrade \hfill 15.1.8061

       \end{large}
  \end{center}

  \begin{center}
    \begin{large}
      Professora - Carla Rodrigues Figueiredo Lara
    \end{large}
  \end{center}

  \vfill

  \begin{center}
    \begin{large}
      João Monlevade \\
      \today \\
    \end{large}
  \end{center}

\clearpage
\tableofcontents 
\end{titlepage}

%--------------------------------------------------------
\section{Adequação ao cronograma}
Na apresentação da proposta, propusemos um cronograma de trabalho que não pôde ser cumprido. Isso se deu ao fato de que o \emph{Middleware} que iríamos utilizar não serviu ao nosso propósito e após muitas pesquisas, vimos ser inviável o uso de arquitetura \emph{peer-to-peer} nesse trabalho de replicação de dados.

\section{Decisões de Projeto}
A primeira e mais importante decisão de projeto tomada foi o abandono da arquitetura \emph{peer-to-pier} e uso de arquitetura Cliente-Servidor. 

Como o fato de não se ter um servidor central é pré-requisito do projeto, nosso trabalho fica concentrado em duas entidades:
\begin{itemize}
\item Banco de IDs: Um mini-servidor, serve para armazenar IDs das operações feitas no Banco de Dados e o nome do respectivo cliente que fez essa operação.
\item Cliente: Apesar do nome, esse cliente é ao mesmo tempo Cliente (na relação com o Banco de IDs) e pode ser Cliente ou Servidor na relação com outros Clientes.
\end{itemize}

Essa comunicação funciona da seguinte forma: 
\begin{enumerate}
\item Cliente que deseja fazer uma alteração no banco de dados local se comunica com o Banco de IDs para que este atualize sua tabela de IDs colocando o nome deste Cliente no respectivo local de sua alteração (O Banco de IDs não armazenará nenhum script de alteração no banco de dados, somente o ID da operação e o nome do cliente que a fez).
\item Todos os Clientes possuem uma \emph{thread} que monitora essa tabela no Banco de IDs via RMI, assim, no momento em que essa thread detecta uma alteração na tabela, os Clientes pegam o nome de quem fez a alteração através de um método remoto.
\item Uma vez em posse do nome de quem fez a alteração, os Clientes desatualizados se comunicam via outra interface RMI com o Cliente que está atualizado e pegam o script da alteração que foi feita no banco.
\item Ao aplicar o mesmo script em seus respectivos Bancos de Dados Locais, os bancos se mantém atualizados, a replicação foi feita com sucesso.
\end{enumerate}

Outra decisão importante tomada foi com relação a arquitetura \emph{Model-View-Controller}, foi decidido que seria mais organizado manter toda interação de rede dentro do pacote \emph{Model} de modo que este teria, então, as regras de negócio, toda a estrutura de controle do banco de dados e ainda toda a interface de comunicação em Rede. 

Dessa forma, fica claro que o pacote \emph{Model} concentra a maior parte do trabalho, deixando o cronograma inicial (com 1 mês de desenvolvimento de cada pacote) obsoleto, uma vez que o \emph{Model} é a parte mais trabalhosa desse projeto.

Dessa forma, por não ter como trazer o pacote \emph{Model} pronto no momento dessa apresentação, decidimos por criar um programa simples de teste dessas interfaces RMI para demonstrar como será seu funcionamento no futuro.

\end{document}
